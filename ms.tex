\documentclass[apj, revtex4]{emulateapj}

\usepackage{epsf}
\usepackage{color}
\usepackage{amsmath}
\usepackage{graphicx}
\usepackage[colorlinks,urlcolor=blue,citecolor=blue,linkcolor=blue]{hyperref}
\usepackage{threeparttable}
\usepackage{multirow}
\usepackage{natbib}
\usepackage{etoolbox}

\bibliographystyle{apj}
\citestyle{aa}
 
% General commands that are good for any astronomy paper.
\input{commands.tex}

\makeatletter
% Patch case where name and year are separated by aysep
\patchcmd{\NAT@citex}
  {\@citea\NAT@hyper@{%
     \NAT@nmfmt{\NAT@nm}%
     \hyper@natlinkbreak{\NAT@aysep\NAT@spacechar}{\@citeb\@extra@b@citeb}%
     \NAT@date}}
  {\@citea\NAT@nmfmt{\NAT@nm}%
   \NAT@aysep\NAT@spacechar\NAT@hyper@{\NAT@date}}{}{}

% Patch case where name and year are separated by opening bracket
\patchcmd{\NAT@citex}
  {\@citea\NAT@hyper@{%
     \NAT@nmfmt{\NAT@nm}%
     \hyper@natlinkbreak{\NAT@spacechar\NAT@@open\if*#1*\else#1\NAT@spacechar\fi}%
       {\@citeb\@extra@b@citeb}%
     \NAT@date}}
  {\@citea\NAT@nmfmt{\NAT@nm}%
   \NAT@spacechar\NAT@@open\if*#1*\else#1\NAT@spacechar\fi\NAT@hyper@{\NAT@date}}


%Commands specific to the this work
\newcommand{\editorial}[1]{\textcolor{red}{#1}}
\DeclareRobustCommand{\ion}[2]{%
\relax\ifmmode
\ifx\testbx\f@series
{\mathbf{#1\,\mathsc{#2}}}\else
{\mathrm{#1\,\mathsc{#2}}}\fi
\else\textup{#1\,{\mdseries\textsc{#2}}}%
\fi}

\newcommand{\citeeg}[1]{(\eg, \citealt{#1})}

%-----------------------------------------------------------------------------------------

\shorttitle{Cluster Dynamics with HETDEX}
\shortauthors{BOADA ET AL.}

%\slugcomment{\it Draft Version \today}
%\slugcomment{\it Submitted for publication in the Astrophysical Journal}
%\slugcomment{Accepted for Publication in the Astrophysical Journal}

\begin{document}

\title{Cluster Dynamics Using the Hobby Eberly Telescope Dark Energy Experiment}

\author{\sc Steven Boada\altaffilmark{1}, 
C.~Papovich\altaffilmark{1}, and
R.~Wechsler\altaffilmark{2,3}} 

\altaffiltext{1}{George P.\ and Cynthia Woods Mitchell Institute for
Fundamental Physics and Astronomy, and Department of Physics and Astronomy,
Texas A\&M University, College Station, TX, 77843-4242;
boada@physics.tamu.edu}
\altaffiltext{2}{Kavli Institute for Particle Astrophysics and Cosmology, Department of Physics, Stanford University, Stanford, CA 94305, USA}
\altaffiltext{3}{Department of Particle Physics and Astrophysics, SLAC National Accelerator Laboratory, Menlo Park, CA 94025, USA}

\begin{abstract}
\noindent
ABSTRACT GOES HERE!!
Lorem ipsum dolor sit amet, consectetur adipisicing elit, sed do eiusmod tempor incididunt ut labore et dolore magna aliqua. Ut enim ad minim veniam, quis nostrud exercitation ullamco laboris nisi ut aliquip ex ea commodo consequat. Duis aute irure dolor in reprehenderit in voluptate velit esse cillum dolore eu fugiat nulla pariatur. Excepteur sint occaecat cupidatat non proident, sunt in culpa qui officia deserunt mollit anim id est laborum.
\end{abstract}

\section{INTRODUCTION}
The Hobby Eberly Telescope Dark Energy eXperiment (HETDEX; \citealt{Hill2008}) is a trailblazing effort to observe high-redshift large scale structures using cutting edge wide-field integral field unit (IFU) spectrographs. Designed to probe the evolution of the dark energy equation of state etched onto high redshift ($z>2$) galaxies by the Baryon Acoustic Oscillations \citep{Eisenstein2005} in the first moments of the universe, the survey will observe two fields for a total of 420 \degsq\ from two fields (300 \degsq, Spring field and 120 \degsq, Fall field). Tuned to find \lya\ emitting (LAE) galaxies at $1.9<z<3.5$, HETDEX expects to find 800,000 LAEs, and more than one million [\ion{O}{ii}] emitting galaxies at $z<0.5$ masquerading as high-redshift galaxies \citep{Acquaviva2014}.  

While a large portion of the $\sim10^6$ interloping [\ion{O}{ii}] galaxies will be field (not associated with a bound structure) galaxies, the large area covered by HETDEX is expected to contain as many as 100 Virgo-sized ($M_{dyn}\sim 10^{15}$ \msol) clusters at $z<0.5$ (\editorial{citation?}). The near-complete spectroscopic coverage allows an unprecedentedly detailed look at a very large number of clusters ranging from group scales to the very massive. In addition to the recovery of accurate dynamical masses, detailed investigations of the of dynamical state of the clusters is possible. 

In addition to the spectroscopy, an imaging campaign is require first identify galaxy over densities. HETDEX overlaps with the Sloan Digital Sky Survey (SDSS; \citealt{Blanton2001a}), SDSS stripe 82 \citep{Annis2014}, the Dark Energy Survey (DES; \citealt{DES2005}), and the upcoming DECam/IRAC Galaxy Environment Survey (DIRGES; PI: Papovich, C. Papovich \etal in preparation).

DIRGES, in particular, will be useful as it will identify as many as 100 galaxy cluster progenitors at 0.8 $<$ z $<$ 1.5 based on similarities in galaxy \spitzer/IRAC colors. Significant gains in the understanding of these clusters are possible if we can (1) interpret their dynamical nature through spectroscopy, and (2) if we can link them to galaxy clusters at lower redshift, z $<$ 0.5, from within the same dataset to suppress potential observational biases. However, two large issues remain.

It is unclear how a blind spectroscopic survey with an IFU will effect the recovery of galaxy cluster dynamical properties. Unlike many previous large cluster surveys \citeeg{Milvang-Jensen2008, Robotham2011, Sifon2015} which use multi-object spectrographs, the Visible Integral-Field Replicable Unit Spectrograph (VIRUS; \citealt{Hill2012}) used by HETDEX samples the sky unevenly which could excluded member galaxies which would otherwise be included. Secondly, it is not straightforward to use spectroscopic redshifts predominately from emission-line galaxies to interpret the kinematic and dynamical states of the clusters.

This work plans to address these concerns in the following ways. Create and evaluate a HETDEX like selection ``function'' of galaxies over a similarly large portion of the sky, use well adopted techniques to recover the dynamical properties, such as velocity dispersion and mass, and to investigate the presence of substructure. This will better allow future work to predict the number and types of galaxy clusters which should be observed with HETDEX.

\editorial{Give outline of paper section.}

Throughout this paper, we adopt the following cosmological model ($\Omega_\Lambda = 0.77$,
$\Omega_M = 0.23$, $\sigma_8 = 0.83$ and $H_0= 72$ \kms \mpc), assume a Chabrier initial mass function (IMF; \citealt{Chabrier2003}), and use AB magnitudes \citep{Oke1974}.

%This support will allow Mr. Boada to contribute to the core science of my NSF grant (AST 1413317) for the DECam/IRAC Galaxy Environment Survey (DIRGES). He will studying both simulated and real data to understand the reliability in measuring galaxy cluster masses and dynamical states using spectroscopic information for emission-line galaxies. DIRGES will identify as many as 100 galaxy cluster progenitors at 0.8 $<$ z $<$ 1.5. Significant gains in the understanding of these clusters are possible if we can (1) interpret their dynamical nature through spectroscopy, and (2) if we can link them to clusters at lower redshift, z $<$ 0.5, from within the same dataset to suppress potential observational biases. I have plans to obtain the spectroscopy to achieve both of these supplemental goals, but the majority of the spectroscopic redshifts will be for emission-line galaxies. It is not straightforward to use these data to interpret the kinematic and dynamical states of the clusters. And, because HETDEX covers DIRGES, the data will contain both low-redshift and high-redshift clusters in the same imaging dataset, allowing the most homogeneous study of the low-z and high-z universe yet.

\section{Data and Observations}

Blah blah intro stuff... 

\subsection{The ``Aardvark'' Catalogs}
\begin{figure} 
	\includegraphics[width=0.49\textwidth]{surveyArea.pdf} 
	\caption{A polar azimuthal equidistant projection of the southern sky showing the sky area (approximately a quarter of the sky) covered by the Aardvark Simulations. Concentric circles are lines of constant declination, 10 to $-80\degree$ in $10\degree$ increments. Radial lines are lines of constant right ascension. Opaqueness of the region indicates the overlap of catalog tiles (see text for details).} 
	\label{fig:survey area} 
\end{figure}

The ``Aardvark'' mock galaxy catalogs (R. Wechsler et al., private communication) are derived from a semi-analytical model (SAM) tied to an in house n-body cosmological simulation and designed for DES. They provide a 10313 \degsq (one quarter of the sky) catalog to full DES depth and contains 1.36 billion galaxies which have a signal-to-noise of at least five in one or more DES observing bands. The catalog footprint is shown in Figure~\ref{fig:survey area}.

A brief description of the catalog creation is described as follows. The initial conditions are generated with a second-order Lagrangian perturbation theory using {\tt2LPTic} \citep{Crocce2006}. Dark matter (DM) n-body simulations are run using {\tt LGadget-2} (a version of {\tt Gadget-2}; \citealt{Springel2005}), and four different light cones are stitched together along the line of sight to create the full light cone. The DM halos are identified using the {\tt ROCKSTAR} halo finder \citep{Behroozi2013} which also calculates halo masses and other various parameters. Galaxies are assigned to halos and subhalos using the ``ADDGals'' method of (R. Wechlar \etal, in preparation) which matches galaxy luminosities with local DM densities, not the individual DM halos. This becomes important in the velocity dispersion and subsequent dynamical mass recovery, see Section~\ref{sec:LOSVD}.

The catalog information, used in this study, is broken into two large portions. The ``truth'' files contain the characteristics of each individual galaxies, such as right ascension (RA), declination (DEC), redshift (z), observed and rest-frame magnitudes, and many others. The ``halo'' files contain information for individual halos, to which many individual galaxies may belong. This includes five estimations of dynamical mass, RA, DEC, z, three dimensional velocity dispersion, and many others.

However, the catalogs do not include information on emission or absorption lines or estimations of whether the halo is relaxed or not. We supplement the catalogs with this information and describe the method in Section~\ref{sec:oii luminosity} and others. 

\subsection{ {\rm[\ion{O}{ii}]} Luminosity}\label{sec:oii luminosity}
The aardvark ``truth'' catalog does not provide [\ion{O}{ii}] luminosities so we must assign them empirically. We use 503113 galaxies from the SDSS Data Release 12 \citep{Alam2015} from $z = 0.05 - 0.2$, which are selected with no redshift warning. 

\begin{figure*} 
	\includegraphics[width=\textwidth]{oii_sdss.pdf} 
	\caption{\textit{Left}: CMD of 503113 $z<0.2$ galaxies take from the SDSS DR12 where the shading scales with the density of points. The two boxes show regions containing potential catalog galaxies. \textit{Right}: Probability histograms of the Log [\ion{O}{ii}] luminosity for the SDSS galaxies located in the two highlighted regions on the right. New [\ion{O}{ii}] luminosity (and subsequently fluxes) are assigned to catalog galaxies from slice sampling the probability histogram.} \label{fig:oii sdss} 
\end{figure*}

While the difference is small, we convert each SDSS galaxy's observed magnitude to a similar DES magnitude using
\begin{eqnarray}
	g_{DES} = g_{SDSS} - 0.104(g-r)_{SDSS} + 0.01 \nonumber \\
	r_{DES} = r_{SDSS} - 0.102(g-r)_{SDSS} + 0.02
\end{eqnarray}
from \cite{Bechtol2015} and place each on a color-magnitude diagram (CMD) of $M_r$ and $g-r$, see Figure~\ref{fig:oii sdss}. To assign an [\ion{O}{ii}] luminosity to each galaxy in our catalog we place the catalog galaxies on the same CMD and select all SDSS galaxies in a small 2D ($M_r$, $g-r$) bin around the galaxy. We extract all of the SDSS galaxies inside that bin and create a histogram of their [\ion{O}{ii}] luminosities. Using a slice sampling technique \citep{Neal1997} we assign the catalog galaxy an [\ion{O}{ii}] luminosity based on the distribution of SDSS galaxies extracted. For catalog galaxies which are placed on the CMD near no, or very few ($1\leq n<10$) galaxies we assign it zero [\ion{O}{ii}] luminosity or the mean luminosity, respectively.

The right panel of Figure~\ref{fig:oii sdss} shows the CMD of all SDSS galaxies. Two potential catalog galaxies are also placed on the CMD ($M_r, g-r = -17.7,~0.49$ and $M_r, g-r = -21.4,~1.24$) and indicated by two colored boxes. The histograms show in the Figure's left panel shows the probability density histograms of the Log [\ion{O}{ii}] luminosity for the SDSS galaxies in the 2D bin. We sample the distribution and assign each catalog galaxy an [\ion{O}{ii}] luminosity which is then converted into a flux.

\subsection{Mock Observations}\label{sec:observations}
\editorial{don't think I say what are the actual selection criteria for the spectroscopy. That's an important detail.}
Tentatively slated to start in the fall of 2015, HETDEX will perform blind spectroscopy (R $\sim$ 750 in $3500 - 5500~\AAA$) over two fields along the celestial equator. The 300 \degsq, Spring field and 120 \degsq, Fall field will have no preselected targets. Using VIRUS on the 10-m Hobby-Eberly Telescope (HET; \citealt{Ramsey1998}) the completed survey is expected to have an overall fill-factor of 1/4.5, meaning that the entire area could be covered with 4.5 dithers of the entire survey. 

The spectral coverage allows for the detection of [\ion{O}{ii}] ($\lambda\lambda 3727-3729~\AAA$ doublet) emitters to $z\sim 0.5$ and Ca H ($\lambda 3968.5~\AAA$) and K ($\lambda 3933.7~\AAA$) absorption features to $z\sim 0.4$. HETDEX is expected to detect sources with continuum brighter than 22 mag in SDSS \sdssg, and emission line strengths above $3.5\times10^{-17}$ \ergscm. So we ``observe'' $z<0.4$ galaxies which meet either the emission line or the magnitude limit. Galaxies $0.4<z<0.5$ are only observed if their emission line strength is sufficient.   

\begin{figure} 
	\includegraphics[width=0.49\textwidth]{f01.pdf} 
	\caption{Representative observation tiling scheme for the HETDEX $16'$ pointings. Each colored square is a single VIRUS IFU and the dashed octagons approximate the size of a single observation. See the text for more details.} \label{fig:ifu layout} 
\end{figure}

Our ``observations'' consist of placing masks down onto the aardvark ``truth'' catalogs and selecting all, $z< 0.5$ also meeting sensitivity limits, galaxies which lie underneath. Each mask is created to accurately reproduce the HETDEX IFU pattern, see Figure~\ref{fig:ifu layout}. The pattern consists of 78 IFUs, which are comprised of 448 optical fibers subtending a $50'' \times 50''$ region on the sky \citep{Kelz2014}. The inter-IFU spacing is also $50''$ spanning a total area of $16'\times 16'$ on the sky. 

The individual IFUs have a fill-factor of 1/3, which will be completely filled with three dithers of the telescope at each pointing. This means that when selecting galaxies from the aardvark catalog we assume an observation for all galaxies laying within a colored, IFU square in Figure~\ref{fig:ifu layout}. Galaxies which lie between the IFUs are missed, as well as the galaxies which lie between the pointings, as there is no overlap between one pointing and the next.

\section{Cluster Membership}
In the following sections we assume that the on sky positions and redshifts of the galaxy clusters are already known. In reality, galaxy over densities will be detected from imaging surveys \citeeg{Rykoff2014} or through techniques to identify structure in the RA, DEC, z point data (\eg, DBSCAN; \citealt{Ester1996}).

While the galaxy cluster center may be well known, the constituent galaxies will not. The identification of member galaxies from photometric data \citeeg{Rozo2014} is promising, however, with spectroscopic information on all galaxies, we choose to utilize the full 3D parameter space. \editorial{Not very happy about that sentence}.

We employ the ``shifting gapped" method of \cite{Fadda1996} as outlined in \cite{Owers2011}, which combines both the positional and velocity information. After an initial cut of $cz_c \pm 10,000$ \kms, the galaxies are binned in $0.4h^{-1}$ Mpc or larger enough bins to contain at least 10 (arbitrarily chosen) galaxies. Galaxies farther than 3 Mpc from the cluster center are also rejected. Once binned, the galaxies are sorted by their line-of-sight velocity (LOSV; see Section~\ref{sec:LOSVD}). Any galaxy with a LOSV greater than 1000 \kms of a neighboring galaxy is rejected as an interloper. The procedure repeats until the number of galaxies stabilizes in the bin.

\editorial{This is taken from the other paper, and needs to be tweaked to make fit better into this framework. I think what we'll do is center a 100'' box on all of the clusters and then try to recover the galaxies. How many will we get, how many are wrong, and can we do better with a different algorithm?}

\section{Recovery of Parameters}
 In the following sections, we outline the methods we use to derive the dynamical properties of the galaxy clusters in our sample. This is not meant to be an exhaustive study of the different methods used to recover these parameters. The following is, in many cases, a subset of the available methods to derive any single parameter. The specific choice of method may improve or diminish the accuracy of the recovered parameter, but the methods chosen were to facilitate comparison with observational studies.  

\subsection{Cluster Redshift}
The accurate determination of the cluster redshift ($z_c$) is crucial to the reliability of all following measurements. An incorrect cluster redshift introduces errors into the measured line of sight velocity (LOSV) and corresponding dispersion, which, in turn, contributes to errors associated with dynamical mass and radius. 

In simple terms, the cluster redshift is the  mean of the redshifts of all galaxies associated with the cluster. However, because the standard mean can be quite sensitive to outliers or otherwise contaminated data, we require a more resistant statistic, and turn to the biweight location estimator \citep{Beers1990} which provides improved performance. 

\subsection{Line of Sight Velocity Dispersion}\label{sec:LOSVD}
We first calculate the line of sight velocity (LOSV) to each galaxy, where
\begin{equation}
	LOSV = c\frac{z - z_c}{1+z_c}
\end{equation}
and $c$ is the speed of light in \kms, $z$ is the redshift of the individual galaxy, and $z_c$ is the overall cluster redshift described in the previous section.

The unbiased estimation of a standard deviation (a measure of statistical dispersion) from these LOSVs is a technically involved problem. At first, we require that our estimator be unbiased in that the dispersion estimation is equal to the true dispersion regardless of the number of points sampled, although, in practice this rarely occurs. The most commonly used estimate is the corrected sample standard deviation, given by:
\begin{equation}
	s = \sqrt{\frac{1}{n-1} \sum_{i=1}^n (x_i - \bar{x})^2}
\end{equation}
with $\{x_1, x_2, ..., x_n\}$ being the random sample and $\bar{x}$ the sample mean. The corrected sample standard deviation has the advantage in that it is unbiased (as opposed to the population standard deviation which is biased), but the removal of bias relies on knowing \textit{a priori} the underlying distribution from which the sample is drawn. An estimator which correctly estimates a standard deviation for a sample drawn from a wide range of distributions and is not adversely effected by outliers is said to be robust. An estimator which correctly identifies the standard deviation, even when a number of points are replaced by different values is said to be resistant.

In our situation, a resistant estimator becomes critical as it minimizes the effect of outliers which the inclusion of non-cluster members have on the measured dispersion. \cite{Beers1990} present both a robust and resistant estimation of scale (yet another term for statistical dispersion) called the biweight scale estimator, which has been widely accepted by the community (\eg, \citealt{Milvang-Jensen2008, Owers2011, Murphy2011} and many others). It is given by 
\begin{equation}
	\sigma_{BI} = \sqrt{ N_{members} \frac{ \sum_{|u_i|<1} (1-u_i^2)^4 (v_i - \bar{v})^2} {D} }
\end{equation}
with $v_i$, the proper velocities, $\bar{v}$, the average of the proper velocities,
\begin{equation}
	D = \sum_{|u_i|<1} (1-u_i^2)(1-5u_i^2)
\end{equation}
with $u_i$ being the biweight weighting and defined as:
\begin{equation}
	u_i = \frac{v_i - \bar{v}}{9 {\rm MAD}(v_i)}
\end{equation}
where MAD is the median absolute deviation. \cite{Ruel2014} note, however, that the biweight scale estimator is biased and suggest a correction of $1/(D-1)$ to $\sigma_{BI}$ referring to it as the biweight sample variance. We adopt this method for clusters with at least 15 members. For clusters with fewer than 15 members, we adopt the gapper scale estimator \citep{Beers1990} which is shown to accurately recover the dispersion for as few as 5 member galaxies \citep{Hou2009}.

There are several situations which can adversely effect the quality of our velocity dispersion measurement. The first is due to the nature of our observations. Because the HETDEX survey has a 1/4.5 fill factor (see Section~\ref{sec:observations}), there will be many situations where we will not have full spectroscopic information for a cluster. Some clusters could also have intrinsic velocity distributions where the observations of only a few members could result in a LOSVD which differs by large percent. This gives rise to clusters along a fixed mass having a large range of possible LOSVDs. \cite{Saro2013} find this effect to be most pronounced for clusters with fewer than 30 members, and \cite{Ruel2014} show the effect is a net increase in scatter of about 5\% compared to clusters with greater than 30 members. \editorial{This section was adapted, pretty heavily from Ruel2014. Is that cool?}

A second effect, unique to works such as this, is the difference between the LOSVD computed between dark matter halos and the galaxies that occupy them, the so-called velocity bias \citeeg{Evrard2008, White2010}. This effect is assumed to be small ($\sim5\%$) and for this work we assume there is no bias, or that the galaxies perfectly trace the velocity distribution.

\subsection{Dynamical Mass}\label{sec:mass}
Recently, the relationship between the LOSVD and dynamical mass has been the focus of several studies \citeeg{Evrard2008, Saro2013, Sifon2013, VanderBurg2014}, and a best fitting relationship for the mass enclosed by $r_{200c}$ of the form
\begin{equation}
	M_{200c} = \frac{10^{15}}{h(z)} \bigg{(}\frac{\sigma_{1D}}{A_{1D}} \bigg{)}^{-\alpha} \Msol
\end{equation}
with $A_{1D} =$ 1040-1140 \kms\ (\citealt{Munari2013}; referred to as $\sigma_{15}$ in \citealt{Evrard2008} and other works), $\alpha = 1/3$, $h(z) = H(z)/100$, and $\sigma_{1D}$ is the LOSVD of the velocity tracers (dark matter particles, subhalos or galaxies).

The choice $A_{1D}$ and $\alpha$ varies between studies \citeeg{Munari2013, VanderBurg2014} and should be calibrated on a individual basis. To do this, we randomly select 47494, $z<0.5$ clusters composed of 36000 $10^{13}$, 6000 $10^{14}$ and two $10^{15}~\Msol$ halos. We perform a linear fit to the $\sigma_{1D}-M_{200}$ values allowing both $A_{1D}$ and $\alpha$ to vary. We find best fitting parameters of $A_{1D} = 1117.9~\kms$ and $\alpha = 0.3297$, both of which are very near the values from \cite{Evrard2008} of $A_{1D} = 1082.9 \pm 4.0~\kms$ and $\alpha = 0.3361$. Therefore, we chose to adopt the parameters from \cite{Evrard2008} to better facilitate with other simulations \citeeg{Old2014}, and observational studies \citeeg{Brodwin2010}.

\subsection{Radial Extent}
We define the radius of a galaxy cluster as
\begin{equation}
	r_{200c} = \bigg{[} \frac{3}{4\pi} \frac{M_{200}}{200\rho_c} \bigg{]}^{1/3}
\end{equation}
where $\rho_c$ is critical density of the universe and defined as $3H^2/8\pi G$.


\section{Evidence of Substructure}
Like the recovery of dynamical parameters discussed in the previous section, the choice of method to investigate the dynamical state of galaxy clusters is heavily debated. Perhaps the most widely used metric is the Dressler-Shectman (DS) test \citep{Dressler1988}, and is not without criticism \citeeg{White2010}. New methods such as the Caustic Method (\citealt{Yu2015}, and the references therein) offer promise to shed further light on this difficult problem.

\subsection{From VD profiles}
We utilize the modality \citep{Oliva-Altamirano2014} which describes the Gaussianity of a LOSV distribution in a galaxy cluster. For clusters with Gaussian distributed LOSVs, the modality is close to 1/3. It is defined as $(1+\mathrm{skewness}^2)/(3+\mathrm{kurtosis}^2)$. \editorial{Do I need to describe what the skewness and kurtosis is?}

\subsection{Dominance}
\editorial{This is taken pretty directly from the GAMA paper. Will need to be edited to fit.}
The dominance is defined as the luminosity gap between the brightest and the second brightest galaxy in a cluster ($\Delta m_{1, 2}$). The amplitude of the luminosity gap between the BGG/BCG and the second brightest galaxy in the halo, is expected to be a function of both the formation epoch and the recent infall history of the halo. A small magnitude gap ($\Delta m_{1, 2} < 1$) indicates a recent halo merger, and larger gaps ($\Delta m_{1, 2} > 1$), common in fossil groups, is perhaps indicative of a cluster or groups that has not undergone a recent merger.

\subsection{Dressler-Schectman Test}
We leverage the large spectroscopic dataset to study the structural properties of the clusters. \cite{Pinkney1996} determine, from a comparison of five different methods that the DS test is the most sensitive to the presence of substructure.

The DS test, which combines the spatial positions and velocities of the galaxies, provides a method to locate substructure by identifying groups of galaxies which differ significantly from the cluster velocity distribution. Galaxy subsets are selected from a cluster of $n_{members}$ and each constituent galaxy deviation is calculated according to
\begin{equation}
	\delta_i^2 = \frac{N_{local}+1}{\sigma^2}\bigg{[}(\bar{v}_{local,i} - \bar{v})^2 + (\sigma_{local,i} - \sigma)^2\bigg{]}^2
\end{equation}
where $\bar{v}_{local}$ and $\sigma_{local}$ are the mean velocity and velocity dispersion for a subset of $N_{local}$ galaxies and $\bar{v}$ and $\sigma$ are the entire cluster's mean velocity and velocity dispersion. The choice of $N_{local}$ is left to the user. Originally, \cite{Dressler1988} choose $N_{local}=10$, however \cite{Bird1994} points out that using a fixed value for $N_{local}$ reduces the sensitivity to substructure. We follow \cite{Bird1994} in choosing $N_{local} = \sqrt n_{members}$. \editorial{doesn't talk about the nearest neighbors which is what we are using.}

The DS statistic is the $\Delta$-value given by, 
\begin{equation}
	\Delta = \sum^{n_{members}}_i \delta_i
\end{equation}
where a system is considered to contain substructure if $\Delta/n_{members} > 1$ \citep{Dressler1988}. A second method, described in \cite{Hou2012}, uses probabilities (P-values) rather than a threshold for the identification of substructure. P-values are computed by comparing the observed $\Delta$-value and the $Delta$-value after the velocities (but not positions) have been shuffled through a series of Monte Carlo runs. The probability of the existence of substructure becomes
\begin{equation}
	P = \sum (\Delta_{shuffled} > \Delta_{Observed}) / n_{shuffle}
\end{equation}
where $n_{shuffle}$ is the number of shufflings used. \editorial{This sounds a whole lot like the description given in Hou2012. Make sure that we aren't copying anything word for word. That'd be bad.}

In practice, we use locate the nearest neighbors using an unsupervised k-nearest neighbor algorithm as implemented in Scikit-Learn \citep{Pedregosa2012}. 

\section{RESULTS}
Our results section is broken into two large sections. First we analyze our ability to recover cluster parameters from an observing strategy similar to that of HETDEX. In the second Section, we apply observational limits, similar to those expected to apply, and investigate how the parameter recovery improves (or worsens) under those conditions. 

\subsection{Full Knowledge}
We compare the calculated cluster redshifts to the true redshift ($z_{c,true}$) for 114903 galaxies comprising 1379 unique halos, and we find a RMS$[\Delta z/(1+z_{c,true})]= 4\times 10^{-4}$ where $\Delta z = z_{c,true} - z_{c}$. \editorial{The RMS using just the regular mean is 3.8e-4 which is only slightly better than the biweight. I don't think that will be the case when we get more complicated things.}

\subsubsection{Dynamical Properties}
\begin{figure*} 
	\includegraphics[width=\textwidth]{fullKnowledge.pdf} 
	\caption{Intrinsic cluster properties versus the recovered cluster properties for 16758 unique halos comprised of 657916 individual galaxies. The left panel shows the line of sight velocity dispersion (LOSVD) and the right panel shows the recovered halo mass. In both panels, the red line shows the 1:1 relation and the teal line shows the best fit to data.} 
	\label{fig:full} 
\end{figure*}
Lorem ipsum dolor sit amet, consectetur adipisicing elit, sed do eiusmod tempor incididunt ut labore et dolore magna aliqua. Ut enim ad minim veniam, quis nostrud exercitation ullamco laboris nisi ut aliquip ex ea commodo consequat. Duis aute irure dolor in reprehenderit in voluptate velit esse cillum dolore eu fugiat nulla pariatur. Excepteur sint occaecat cupidatat non proident, sunt in culpa qui officia deserunt mollit anim id est laborum.

\subsubsection{Structural Properties}
Lorem ipsum dolor sit amet, consectetur adipisicing elit, sed do eiusmod tempor incididunt ut labore et dolore magna aliqua. Ut enim ad minim veniam, quis nostrud exercitation ullamco laboris nisi ut aliquip ex ea commodo consequat. Duis aute irure dolor in reprehenderit in voluptate velit esse cillum dolore eu fugiat nulla pariatur. Excepteur sint occaecat cupidatat non proident, sunt in culpa qui officia deserunt mollit anim id est laborum.

\subsection{HETDEX Depth}

\subsubsection{Dynamical Properties}
\begin{figure*} 
	\includegraphics[width=\textwidth]{hetdexDepth.pdf} 
	\caption{HETDEX DEPTH! Intrinsic cluster properties versus the recovered cluster properties for 14239 unique halos comprised of 87754 individual galaxies. The left panel shows the line of sight velocity dispersion (LOSVD) and the right panel shows the recovered halo mass. In both panels, the red line shows the 1:1 relation and the teal line shows the best fit to data.} 
	\label{fig:hetdexDepth} 
\end{figure*}
Lorem ipsum dolor sit amet, consectetur adipisicing elit, sed do eiusmod tempor incididunt ut labore et dolore magna aliqua. Ut enim ad minim veniam, quis nostrud exercitation ullamco laboris nisi ut aliquip ex ea commodo consequat. Duis aute irure dolor in reprehenderit in voluptate velit esse cillum dolore eu fugiat nulla pariatur. Excepteur sint occaecat cupidatat non proident, sunt in culpa qui officia deserunt mollit anim id est laborum.

\subsubsection{Structural Properties}
Lorem ipsum dolor sit amet, consectetur adipisicing elit, sed do eiusmod tempor incididunt ut labore et dolore magna aliqua. Ut enim ad minim veniam, quis nostrud exercitation ullamco laboris nisi ut aliquip ex ea commodo consequat. Duis aute irure dolor in reprehenderit in voluptate velit esse cillum dolore eu fugiat nulla pariatur. Excepteur sint occaecat cupidatat non proident, sunt in culpa qui officia deserunt mollit anim id est laborum.


\section{IMPROVEMENT}
Lorem ipsum dolor sit amet, consectetur adipisicing elit, sed do eiusmod tempor incididunt ut labore et dolore magna aliqua. Ut enim ad minim veniam, quis nostrud exercitation ullamco laboris nisi ut aliquip ex ea commodo consequat. Duis aute irure dolor in reprehenderit in voluptate velit esse cillum dolore eu fugiat nulla pariatur. Excepteur sint occaecat cupidatat non proident, sunt in culpa qui officia deserunt mollit anim id est laborum.

\section{SUMMARY}
Lorem ipsum dolor sit amet, consectetur adipisicing elit, sed do eiusmod tempor incididunt ut labore et dolore magna aliqua. Ut enim ad minim veniam, quis nostrud exercitation ullamco laboris nisi ut aliquip ex ea commodo consequat. Duis aute irure dolor in reprehenderit in voluptate velit esse cillum dolore eu fugiat nulla pariatur. Excepteur sint occaecat cupidatat non proident, sunt in culpa qui officia deserunt mollit anim id est laborum.

\acknowledgments 
The authors also wish to thank the anonymous referee whose comments and
suggestions significantly improved both the quality and clarity of this work. Several open source resources are used to complete this study Python \citep{van1991} along with Matplotlib \citep{Hunter2007} and IPython \citep{Perez2007}.

\bibliography{master}

\end{document}