\documentclass[apj, revtex4]{emulateapj}

\usepackage{epsf}
\usepackage{color}
\usepackage{amsmath}
\usepackage{graphicx}
\usepackage[colorlinks,urlcolor=blue,citecolor=blue,linkcolor=blue]{hyperref}
\usepackage{threeparttable}
\usepackage{multirow}
\usepackage{natbib}
\usepackage{etoolbox}

\bibliographystyle{apj}
\citestyle{aa}
 
% General commands that are good for any agronomy paper.
%%% Fields %%%
\newcommand{\hdf}{HDF-N}
\newcommand{\hdfn}{HDF-N}
\newcommand{\hdfs}{HDF-S}
\newcommand{\cdfs}{CDF-S}

%%% Telescopes %%%
\newcommand{\hst}{\textit{HST}}
\newcommand{\iras}{\textit{IRAS}}
\newcommand{\iso}{\textit{ISO}}
\newcommand{\spitzer}{\textit{Spitzer}}
\newcommand{\sirtf}{\textit{Spitzer}}
\newcommand{\chandra}{\textit{Chandra}}

%%% Filters %%%
\newcommand{\wfu}{\hbox{$\mathrm{U}_{300}$}}
\newcommand{\wfb}{\hbox{$\mathrm{B}_{450}$}}
\newcommand{\wfv}{\hbox{$\mathrm{V}_{606}$}}
\newcommand{\wfi}{\hbox{$\mathrm{I}_{814}$}}
\newcommand{\acsb}{\hbox{$\mathrm{B}_{435}$}}
\newcommand{\acsv}{\hbox{$\mathrm{V}_{606}$}}
\newcommand{\acsi}{\hbox{$i_{775}$}}
\newcommand{\acsz}{\hbox{$z_{850}$}}
\newcommand{\nicj}{\hbox{$\mathrm{J}_{110}$}}
\newcommand{\nich}{\hbox{$\mathrm{H}_{160}$}}
\newcommand{\wfcy}{\hbox{$\mathrm{Y}_{105}$}}
\newcommand{\wfcj}{\hbox{$\mathrm{J}_{125}$}}
%\newcommand{\wfcj}{\hbox{$J_{110}$}}
\newcommand{\wfch}{\hbox{$\mathrm{H}_{160}$}}
\newcommand{\sdssu}{\hbox{$u$}}
\newcommand{\sdssg}{\hbox{$g$}}
\newcommand{\sdssr}{\hbox{$r$}}
\newcommand{\sdssi}{\hbox{$i$}}
\newcommand{\sdssz}{\hbox{$z$}}
\newcommand{\mone}{\hbox{$[3.6]$}}
\newcommand{\mtwo}{\hbox{$[4.5]$}}
\newcommand{\mthree}{\hbox{$[5.8]$}}
\newcommand{\mfour}{\hbox{$[8.0]$}}
%\newcommand{\mone}{\hbox{$[3.6\mu\mathrm{m}]$}}
%\newcommand{\mtwo}{\hbox{$[4.5\mu\mathrm{m}]$}}
%\newcommand{\mthree}{\hbox{$[5.8\mu\mathrm{m}]$}}
%\newcommand{\mfour}{\hbox{$[8.0\mu\mathrm{m}]$}}

%%% Astronomy Abreviations %%%
\newcommand{\mstar}{\hbox{$\mathrm{M}^\ast$}}
\newcommand{\lstar}{\hbox{$L^\ast$}}
\newcommand{\Msol}{\hbox{$\mathrm{M}_\odot$}}
\newcommand{\msol}{\hbox{$\mathrm{M}_\odot$}}
\newcommand{\Zsol}{\hbox{$Z_\odot$}}
\newcommand{\zsol}{\hbox{$Z_\odot$}}
\newcommand{\Lsol}{\hbox{$L_\odot$}}
\newcommand{\lsol}{\hbox{$L_\odot$}}
\newcommand{\lir}{\hbox{$L_{\mathrm{IR}}$}}
\newcommand{\zph}{\hbox{$z_\mathrm{ph}$}}
\newcommand{\zphot}{\hbox{$z_\mathrm{ph}$}}
\newcommand{\lbol}{\hbox{$L_\mathrm{bol}$}}
\newcommand{\snr}{\hbox{$\mathrm{S/N}$}}
\newcommand{\reff}{\hbox{$r_\mathrm{eff}$}}
\newcommand{\ks}{\hbox{$K_s$}}
\newcommand{\AAA}{\hbox{\AA}}

%%% Spectrum Lines %%%
\newcommand{\lya}{ Ly$\alpha \;$}
\newcommand{\lyb}{Lyman~$\beta$}
\newcommand{\hb}{\hbox{H$\beta$}}
\newcommand{\ha}{\hbox{H$\alpha$}}
\newcommand{\paa}{\hbox{Pa$\alpha$}}

%%% Units %%%
\newcommand{\kms}{\hbox{km~s$^{-1}$}}
\newcommand{\cms}{\hbox{cm~s$^{-1}$}}
\newcommand{\mpc}{\hbox{Mpc$^{-1}$}}
\newcommand{\mpcsq}{\hbox{Mpc$^{-2}$}}
\newcommand{\mpccu}{\hbox{Mpc$^{-3}$}}
\newcommand{\cnts}{\hbox{cnt~s$^{-1}$}} 
\newcommand{\cmsq}{\hbox{cm$^{-2}$}}
\newcommand{\cmcu}{\hbox{cm$^{-3}$}}
\newcommand{\ergscm}{\hbox{erg~s$^{-1}$~cm$^{-2}$}}
\newcommand{\uJy}{\hbox{$\mu$Jy}}
\newcommand{\ujy}{\hbox{$\mu$Jy}}
\newcommand{\degree}{\hbox{$^\circ$}}
\newcommand{\degsq}{\hbox{degree$^2$}}
\newcommand{\arcminsq}{\hbox{arcmin$^2$}}
\newcommand{\um}{\hbox{$\mu$m}}

%%% Math %%%
\newcommand{\lsim}{\lesssim}
\newcommand{\gsim}{\gtrsim}
\newcommand{\mathS}{\hbox{$\mathcal{S}$}}
\newcommand{\mathR}{\hbox{$\mathcal{R}$}}
\newcommand{\mathM}{\hbox{$\mathcal{M}$}}
\newcommand{\mcal}{\hbox{$\mathcal{M}$}}
\newcommand{\rcal}{\hbox{$\mathcal{R}$}}
\newcommand{\scal}{\hbox{$\mathcal{S}$}}
\newcommand{\infinity}{\hbox{$\infty$}}
\newcommand{\err}[2]{$^{+#2}_{-#1}$}

%%% General %%%
\newcommand{\etal}{et al.}
\newcommand{\eg}{e.g.}
\newcommand{\ie}{i.e.}
\newcommand{\cf}{cf.}
% \newcommand{\ion}[2]{\hbox{#1$\;${\small\rm{#2}}}}
\newcommand{\mybullet}{\noindent$\bullet$}
\newcommand{\uit}{\textit{UIT}}
\newcommand{\nd}{...}
%\newcommand{\cmodel}{\hbox{\tt cmodel}}
%\newcommand{\bs}{\hbox{$\!\!\!\!$}}
\newcommand{\todo}[1]{{\tt #1}}
\newcommand{\citeeg}[1]{(\eg, \citealt{#1})}
\newcommand{\ignore}[1]{}

%%% Extra %%% 
% \newcommand{\farcm}{\mbox{\ensuremath{.\mkern-4mu^\prime}}}%fractional arcminute symbol 0.'0
% \newcommand{\farcs}{\mbox{\ensuremath{.\!\!^{\prime\prime}}}}%fractional arcsecond symbol: 0.''0
% \newcommand{\fdg}{\mbox{\ensuremath{.\!\!^\circ}}}%fractional degree symbol:     0.°0
\newcommand{\arcdeg}{\ensuremath{^{\circ}}}%                    % degree symbol:  °
% \newcommand{\sun}{\ensuremath{\odot}}%                          % sun symbol
% \newcommand{\apj}{ApJ}%                                         % Journal abbreviations
% \newcommand{\apjs}{ApJS}
% \newcommand{\apjl}{ApJL}
% \newcommand{\aap}{A{\&}A}
% \newcommand{\aaps}{A{\&}AS}
% \newcommand{\mnras}{MNRAS}
% \newcommand{\aj}{AJ}
% \newcommand{\araa}{ARAA}
% \newcommand{\pasp}{PASP}
\newcommand{\Teff}{\ensuremath{T_{\mathrm{eff}}}}%              % T_eff
\newcommand{\logg}{\ensuremath{\log g}}%                        % log g
\newcommand{\bv}{\ensuremath{B\!-\!V}}%                         % B-V
\newcommand{\ub}{\ensuremath{U\!-\!B}}%                         % U-B
\newcommand{\vr}{\ensuremath{V\!-\!R}}%                         % V-R
\newcommand{\ur}{\ensuremath{U\!-\!R}}%                         % U-R

\makeatletter
% Patch case where name and year are separated by aysep
\patchcmd{\NAT@citex}
  {\@citea\NAT@hyper@{%
     \NAT@nmfmt{\NAT@nm}%
     \hyper@natlinkbreak{\NAT@aysep\NAT@spacechar}{\@citeb\@extra@b@citeb}%
     \NAT@date}}
  {\@citea\NAT@nmfmt{\NAT@nm}%
   \NAT@aysep\NAT@spacechar\NAT@hyper@{\NAT@date}}{}{}

% Patch case where name and year are separated by opening bracket
\patchcmd{\NAT@citex}
  {\@citea\NAT@hyper@{%
     \NAT@nmfmt{\NAT@nm}%
     \hyper@natlinkbreak{\NAT@spacechar\NAT@@open\if*#1*\else#1\NAT@spacechar\fi}%
       {\@citeb\@extra@b@citeb}%
     \NAT@date}}
  {\@citea\NAT@nmfmt{\NAT@nm}%
   \NAT@spacechar\NAT@@open\if*#1*\else#1\NAT@spacechar\fi\NAT@hyper@{\NAT@date}}


%Commands specific to the this work
\newcommand{\editorial}[1]{\textcolor{red}{#1} }
\DeclareRobustCommand{\ion}[2]{%
\relax\ifmmode
\ifx\testbx\f@series
{\mathbf{#1\,\mathsc{#2}}}\else
{\mathrm{#1\,\mathsc{#2}}}\fi
\else\textup{#1\,{\mdseries\textsc{#2}}}%
\fi}

%-----------------------------------------------------------------------------------------

\shorttitle{Cluster Paper}
\shortauthors{BOADA ET AL.}

%\slugcomment{\it Draft Version \today}
%\slugcomment{\it Submitted for publication in the Astrophysical Journal}
%\slugcomment{Accepted for Publication in the Astrophysical Journal}

\begin{document}

\title{Cluster Paper}

\author{\sc Steven Boada\altaffilmark{1}, 
C.~Papovich\altaffilmark{1},
B.~Salmon\altaffilmark{1}, 
N.~Merhtens\altaffilmark{1},
T.~Li\altaffilmark{1}, 
R.~Wechsler\altaffilmark{2,3},
E.~Rozo\altaffilmark{2,4},
E.~Rykoff\altaffilmark{2,5}} 

\altaffiltext{1}{George P.\ and Cynthia Woods Mitchell Institute for
Fundamental Physics and Astronomy, and Department of Physics and Astronomy,
Texas A\&M University, College Station, TX, 77843-4242;
boada@physics.tamu.edu}
\altaffiltext{2}{Kavli Institute for Particle Astrophysics and Cosmology, Department of Physics, Stanford University, Stanford, CA 94305, USA}
\altaffiltext{3}{Department of Particle Physics and Astrophysics, SLAC National Accelerator Laboratory, Menlo Park, CA 94025, USA}
\altaffiltext{4}{Department of Astronomy, University of Chicago, Chicago, IL 60637, USA}
\altaffiltext{5}{E. O. Lawrence Berkeley National Lab, 1 Cyclotron Road, Berkeley, CA 94720, USA}

\begin{abstract}
\noindent
ABSTRACT GOES HERE!!
Lorem ipsum dolor sit amet, consectetur adipiscing elit. Cras malesuada, ante eu scelerisque gravida, augue lacus ullamcorper metus, ac volutpat libero magna rutrum nisi. Interdum et malesuada fames ac ante ipsum primis in faucibus. Ut eget magna est. Nam quis maximus purus, feugiat facilisis lectus. Nulla mollis sodales felis a ullamcorper. Lorem ipsum dolor sit amet, consectetur adipiscing elit. Curabitur rhoncus, ligula vel venenatis ornare, nisi massa sagittis orci, in hendrerit purus libero id mauris. Class aptent taciti sociosqu ad litora torquent per conubia nostra, per inceptos himenaeos.
\end{abstract}

\section{INTRODUCTION}

\section{AARDVARK SIMULATION}

\section{THE HETDEX SURVEY}

\section{ANALYSIS}

\subsection{Mock Observations}

\subsection{Dynamical Properties}
The unbiased estimation of a standard deviation (a measure of statistical dispersion) is a technically involved problem. At first, we require that our estimator be unbiased in that the dispersion estimation is equal to the true dispersion, although, in practice this rarely occurs. The most commonly used estimate is the corrected sample standard deviation, given by:
\begin{equation}
	s = \sqrt{\frac{1}{n-1} \sum_{i=1}^n (x_i - \bar{x})^2}
\end{equation}
with $\{x_1, x_2, ..., x_n\}$ is the random sample and $\bar{x}$ is the sample mean. The corrected sample standard deviation has the advantage in that it is unbiased (as opposed to the population standard deviation which is biased), but the removal of bias relies on knowing \textit{a priori} the underlying distribution from which the sample is drawn. An estimator which correctly estimates a standard deviation for a sample drawn from a wide range of distributions and is not adversely effected by outliers is said to be \textit{robust}.

A robust estimator becomes critical as it minimizes the effect that outliers which may remain after the rejection of non-cluster members have on the measured dispersion. \cite{Beers1990} present a robust estimation of scale (yet another term for statistical dispersion) called the biweight scale estimator. \cite{Ruel2014} note that the biweight scale estimator is biased and suggests a correction. 

The biweight scale estimator is given by \editorial{This isn't very good and needs to be rewritten but it gets the idea across.}:
\begin{equation}
	\sigma_{BI} = \sqrt{ N_{members} \frac{ \sum_{|u_i|<1} (1-u_i^2)^4 (v_i - \bar{v})^2} {D} }
\end{equation}
with $v_i$, the proper velocities, $\bar{v}$, the average of the proper velocities,
\begin{equation}
	D = \sum_{|u_i|<1} (1-u_i^2)(1-5u_i^2)
\end{equation}
with $u_i$ being the biweight weighting and defined as:
\begin{equation}
	u_i = \frac{v_i - \bar{v}}{9 {\rm MAD}(v_i)}
\end{equation}
where MAD is the median absolute deviation.

\editorial{Need to talk about how the new biweight is different. Also need to talk about how we are using DM halos and not galaxies, so we are only going to do so well when comparing the 3d dispersions and the dispersions that we are calculating.}

For the mass \editorial{Following \cite{Evrard2008, VanderBurg2014, Sifon2013}}.

\subsection{Evidence of Substructure}
We leverage the large spectroscopic dataset to study the structural properties of the clusters. \cite{Pinkney1996} determine, from a comparison of five different methods that the Dressler-Shectman (DS) test \citep{Dressler1988} is the most sensitive to the presence of substructure.

The DS test, which combines the spatial positions and velocities of the galaxies, provides a method to locate substructure by identifying groups of galaxies which differ significantly from the cluster velocity distribution. Galaxy subsets are selected from a cluster of $n_{members}$ and each constituent galaxy deviation is calculated according to
\begin{equation}
	\delta_i^2 = \frac{N_{local}+1}{\sigma^2}\bigg{[}(\bar{v}_{local,i} - \bar{v})^2 + (\sigma_{local,i} - \sigma)^2\bigg{]}^2
\end{equation}
where $\bar{v}_{local}$ and $\sigma_{local}$ are the mean velocity and velocity dispersion for a subset of $N_{local}$ galaxies and $\bar{v}$ and $\sigma$ are the entire cluster's mean velocity and velocity dispersion. The choice of $N_{local}$ is left to the user. Originally, \cite{Dressler1988} choose $N_{local}=10$, however \cite{Bird1994} points out that using a fixed value for $N_{local}$ reduces the sensitivity to substructure. We follow \cite{Bird1994} in choosing $N_{local} = \sqrt n_{members}$. \editorial{doesn't talk about the nearest neighbors which is what we are using.}

The DS statistic is the $\Delta$-value given by, 
\begin{equation}
	\Delta = \sum^{n_{members}}_i \delta_i
\end{equation}
where a system is considered to contain substructure if $\Delta/n_{members} > 1$ \citep{Dressler1988}. A second method, described in \cite{Hou2012}, uses probabilities (P-values) rather than a threshold for the identification of substructure. P-values are computed by comparing the observed $\Delta$-value and the $Delta$-value after the velocities (but not positions) have been shuffled through a series of Monte Carlo runs. The probability of the existence of substructure becomes
\begin{equation}
	P = \sum (\Delta_{shuffled} > \Delta_{Observed}) / n_{shuffle}
\end{equation}
where $n_{shuffle}$ is the number of shufflings used. \editorial{This sounds a whole lot like the description given in Hou2012. Make sure that we aren't copying anything word for word. That'd be bad.}

In practice, we use locate the nearest neighbors using an unsupervised k-nearest neighbor algorithm as implemented in Scikit-Learn \citep{Pedregosa2012}. 

%\subsection{Membership}
%It is expected that some of the galaxies detected will be spurious sources not associated with the any given cluster. To reject these interlopers we employ two methods. For clusters with 20 or more $Q=0$ redshifts we use the  ``shifting gapped" method of \cite{Fadda1996}. This procedure combines both the positional and velocity information. Galaxies are binned in $0.4h^{-1}$ Mpc or larger enough bins to contain at least 10 galaxies. The galaxies in each bin are sorted by their line-of-sight velocity (LOSV),
%\begin{equation}
%	v = \frac{c (z-z_{cl})}{(1+z_{cl})}
%\end{equation}
%where $z$ is the redshift of the individual galaxy and $z_{cl}$ is the average redshift of the cluster. Any galaxy with a LOSV greater than 1000 \kms of a neighboring galaxy is rejected as an interloper. The procedure repeats until the number of galaxies stabilizes in the bin.

%For galaxy clusters with fewer than 20 $Q=0$ redshifts we employ the method outlined in \cite{Connelly2012,Wilman2005} where we assume an initial velocity dispersion of 500\kms\ and apply both redshift and spatial limits given by:
%\begin{equation}
%	\delta(r)_{max} = \frac{1}{10}\frac{\delta(z)_{max}}{h_{75}^{-1}\mathrm{Mpc}}
%\end{equation}
%and
%\begin{equation}
%	\delta(\theta)_{max} = 206265'' \frac{\delta(r)_{max}}{h_{75}^{-1}\mathrm{Mpc}} \bigg{(}\frac{D_{\theta}}{h_{75}^{-1}\mathrm{Mpc}} \bigg{)}.
%\end{equation}
%where $\delta(z)_{max} = 2\sigma(v)/c$ and $D_\theta$ is the angular diameter distance.

%The observed velocity dispersion, $\sigma_v$, of each cluster is calculated in two different ways. For clusters with at least 15 confirmed members, we use the biweight scale estimator \citep{Beers1990} which is less susceptible to outliers which may remain after the rejection of non-cluster members.

%For galaxy clusters with fewer than 15 members, we employ the gapper estimator \citep{Beers1990} which provides accurate dispersions for groups as small as 5 members \citep{Hou2009}. The gapped estimator is given by
%\begin{equation}
%	\sigma_v = 1.135c \frac{\sqrt{\pi}}{n(n-1)} \sum_{i=1}^{n-1} w_i g_i
%\end{equation}
%where the weights are $w_i = i(n-i)$ and the gaps $g_i = z_{i+1} -z_i$. The factor of 1.135 is from \cite{Wilman2005} which corrects for the $2\sigma$ redshift space cut applied during membership determination.


\section{RESULTS}

\section{IMPROVEMENT}
\section{SUMMARY}

\bibliography{master}

\end{document}